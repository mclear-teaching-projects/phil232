% Created 2022-01-25 Tue 10:48
% Intended LaTeX compiler: xelatex
\documentclass[12pt]{article}
                           
                           \input{/Users/roambot/.emacs.d/.local/custom-org-latex-classes/notes-setup-file.tex}
\author{PHIL 232}
\date{January 25, 2022}
\title{The Primary/Secondary Quality Distinction}
\makeatletter
\newcommand{\citeprocitem}[2]{\hyper@linkstart{cite}{citeproc_bib_item_#1}#2\hyper@linkend}
\makeatother

\usepackage[notquote]{hanging}
\begin{document}

\maketitle
\tableofcontents

\noindent\rule{\textwidth}{0.5pt}

\section{Background}
\label{sec:orgbe42928}
Aristotle distinguishes between primary and secondary qualities in his
discussion of generation and change (\emph{On Generation and Corruption}, bk.
II). He argues that we should distinguish between qualities that are
responsible for or otherwise explain the coming-to-be and passing-away
of things (primary qualities) and qualities which are distinct from
these, which are features of things, and whose character is explained by
appeal to relationships among the primary qualities.

Primary qualities for Aristotle include hot, cold, wet, dry. These
qualities are the basic explanatory building blocks for things.
Relationships amongst these qualities define the four basic
elements---viz. earth, air, fire, and water---out of which everything
was thought to be made.

The Aristotelian distinction understood the primary qualities as
qualities that were manifest to our senses, and thus understood the
physical world---nature---as the kind of thing whose fundamental
features were primarily intelligible via the senses. This would all
change by the seventeenth century.

\section{Mechanism \& Materialism}
\label{sec:org9bd3549}
The seventeenth century saw the rise of two distinct but related
positions: \emph{mechanism} and \emph{materialism}. \emph{Materialism} is a
metaphysical theory concerning the natural world. It claims that the
natural world consists fundamentally of a kind of \emph{thing} or \emph{stuff}---matter---and
that this stuff, organized in various ways according to natural laws as
characterized by physics, determines the nature and features of all the
objects we experience as part of the objective (and thus
mind-independent) natural world. Materialism says that there is
fundamentally only one kind of thing---matter. Materialism thus stands
in opposition to a \emph{dualist} view like that articulated by Descartes.
Descartes argues that there are fundamentally \emph{two} kinds of
thing---matter, and mind---and that the natural world consists of these
two kinds of thing interacting with one another, sometimes in very
special ways (as with the ensouled bodies of human beings).

\emph{Mechanism} is part of a theory of \emph{explanation}. Lisa Downing, a
scholar of the Early Modern period, puts it this way,

\begin{quote-b}
[Mechanistic doctrine] states that all macroscopic bodily phenomena
should be explained in terms of the motions and impacts of
submicroscopic particles, or corpuscles, each of which can be fully
characterized in terms of a strictly limited range of (primary)
properties: size, shape, motion (or mobility), and, perhaps, solidity
or impenetrability. (\citeprocitem{3}{Downing 1998, 381})
\end{quote-b}

A 'corpuscle' is a extremely small parcel of matter with a determinate size, shape,
motion, location, etc. It is the features of individual corpuscles, plus their
interactions, which mechanism takes as sufficient for explaining the characteristics
and behavior of the natural (material) world. Thus, while the claim that the natural
world consists of nothing but corpuscles, or material particles, is a metaphysical
claim about what there is, mechanism is an explanatory claim concerning how appeal to
microscopic particles and their features is sufficient for explaining all natural
phenomena. These views thus compliment one another and often go together.

Locke himself sometimes frames his understanding of mechanism in terms
of what he calls the 'corpuscularian hypothesis'.

\begin{quote-b}
I have here instanced in the corpuscularian hypothesis, as that which
is thought to go furthest in an intelligible explication of those
qualities of bodies; and I fear the weakness of human understanding is
scarce able to substitute another, which will afford us a fuller and
clearer discovery of the necessary connexion and coexistence of the
powers which are to be observed united in several sorts of [bodies].
(IV.iii.16)
\end{quote-b}

There is a great deal to be said about both the doctrine of materialism
and that of mechanism, as well as their development and influence in the
seventeenth and eighteenth centuries. But here I focus on just one
issue---the specific properties corpuscles were thought to have, and the
explanatory role of these properties relative to all the other apparent
characteristics of objects.

Notice two things about mechanism. First, it ultimately concerns
unobservable, or at least, \emph{unobserved} entities (corpuscles). This
means that the explanation of observable phenomena depends on the
existence and characteristics of unobserved phenomena. This, in and of
itself, is a dramatically anti-Aristotelian move. The fundamental
explanatory level of reality is one which is removed, and perhaps
ineluctably so, from our direct apprehension in experience.

Second, the explanatorily relevant features of corpuscles are taken by
mechanists to be their \emph{geometric} properties---viz., size, shape,
location, and state of motion. What are not included are those features
which we might think of as tied to specific ways of sensing the
world---e.g. their colors, tastes, or smells.\footnote{One feature---solidity---may or may not occupy a special role,
being the only proper sensible (sensory quality tied to a
specific sense modality, in this case touch) that may yet be
explanatorily fundamental for the mechanist. We'll return to this
issue later when discussing Locke and Descartes's theories of
matter.} This bifurcation in
explanatory role meant both that greater emphasis was placed on our
\emph{mathematical} understanding of the natural world, and that our purely
sensory grasp of the natural world no longer played a significant role
(perhaps no role at all) in telling us how and why the world appears to
us as it does. To see this this shift to the characteristically modern
version of the ``primary/secondary'' quality distinction at work, it helps
to start with Galileo.

\section{Galileo on Real Qualities}
\label{sec:org9689db8}
\subsection{Background: Galileo (15 February 1564 -- 8 January 1642)}
\label{sec:org8d603da}
\href{http://plato.stanford.edu/entries/galileo/}{Galileo Galilee} was an Italian physicist, mathematician, astronomer, and philosopher
who played a major role in the Scientific Revolution. His achievements include
improvements to the telescope and consequent astronomical observations and support
for \href{http://plato.stanford.edu/entries/copernicus/}{Copernicanism}.

His contributions to observational astronomy include the telescopic
confirmation of the phases of Venus, the discovery of the four largest
satellites of Jupiter (named the Galilean moons in his honour), and the
observation and analysis of sunspots. Galileo also worked in applied
science and technology, inventing an improved military compass and other
instruments.

Galileo's championing of heliocentrism (\href{http://bit.ly/1RvESfA}{remember?}) was controversial within his
lifetime, when most subscribed to some form of the Ptolemaic geocentric system. He
met with opposition from astronomers, who doubted heliocentrism for various reasons.
The matter was investigated by the Roman Inquisition in 1615, and they concluded that
it could be supported as only a possibility, not an established fact. Galileo later
defended his views in his \emph{Dialogue Concerning the Two Chief World Systems}, which
appeared to attack Pope Urban VIII and thus alienated Galileo from the pope and the
Jesuits, who had both supported him up until this point. In 1633 he was tried by the
Inquisition, found ``vehemently suspect of heresy'', forced to recant, and spent the
rest of his life under house arrest.

\subsection{The \emph{Assayer}}
\label{sec:org7a56f5e}
The \emph{Assayer} was written in 1623 as a defense of Galileo's scientific and
philosophical views against Jesuit criticism. Specifically, it was written in
response to Jesuit professor Orazio Grassi, whose \emph{Astronomical Balance} of 1619
criticized various aspects of Galileo's views concerning astronomy and physics.
It particularly pleased the new pope, Urban VIII, to whom it had been dedicated.
Galileo's dispute with Grassi permanently alienated many of the Jesuits who had
previously been sympathetic to his ideas, and Galileo and his friends were
convinced that these Jesuits were responsible for bringing about his later
condemnation, though the evidence for this is not conclusive (see
(\citeprocitem{2}{Blackwell and Foscarini 1991}; \citeprocitem{1}{Blackwell 2008}).

To ``assay'' is to examine something in order to assess its nature,
particularly in respect to testing a metal or ore in order to determine
its quality or worth. So the title of Galileo's works references the
idea of examination of worth in its own examination of the physical and
astronomical doctrines put forward by Grassi.

\subsubsection{Galileo's Scientific Rationalism \& the Conceivability Argument}
\label{sec:org44f3ab8}
In the \emph{Assayer} Galileo argues that we should distinguish between two
different classes of feature which a material body might have, and that
only one of these classes is relevant to providing explanations of the
nature and behavior of physical phenomena.

\begin{quote-b}
Accordingly, I say that as soon as I conceive of a corporeal substance
or material, I feel indeed drawn by the necessity of also conceiving
that it is bounded and has this or that shape; that it is large or
small in relation to other things; that it is in this or that location
and exists at this or that time; that it moves or stands still; that
it touches or does not touch another body; and that it is one, a few,
or many. Nor can I, by any stretch of the imagination, separate it
from these conditions. However, my mind does not feel forced to regard
it as necessarily accompanied by such conditions as the following:
that it is white or red, bitter or sweet, noisy or quiet, and
pleasantly or unpleasantly smelling; on the contrary, if we did not
have the assistance of our senses, perhaps the intellect and the
imagination by themselves would never conceive of them. Thus, from the
point of view of the subject in which they seem to inhere, these
tastes, odors, colors, etc., are nothing but empty names; rather they
inhere only in the sensitive body, such that if one removes the
animal, then all these qualities are taken away and annihilated.
(\citeprocitem{5}{Galilei 2008, 185})
\end{quote-b}

We might formalize his argument as follows:

\begin{enumerate}
\item Thinking of a material object requires thinking of it as having a
particular set of qualities (i.e. size, shape, location, motion,
etc.)
\item Thinking of a material object does not require thinking of it as
having specific sensory qualities (e.g. color, taste, smell, sound)
\item The only qualities possessed by a material object are those which we
must attribute to it in thought
\item \(\therefore\) Sensory qualities are not qualities of material objects, but rather
features of our consciousness of those objects --- i.e. ``sensations''
[from 1-3]
\item \(\therefore\) If there were no conscious beings there would be no sensory
qualities (e.g. colors, tastes, smells, etc.). [from 4]
\end{enumerate}

Galileo thus argues from the conceivable separation of the geometric
qualities of objects from the 'proper sensibles' such as color and
smell, to the fundamentality of the geometric qualities and the ultimate
unreality (or subject-dependence) of the purely sensory qualities.

Central to this argument is the assumption which lies behind premise
(3), that the only qualities we need attribute to a material being are
those which we must attribute to it in thought. This assumes what we can
call '\emph{modal rationalism}',

\begin{description}
\item[{Modal Rationalism:}] The actual, possible, and necessary features of
reality may be determined by examining the actual, possible, and
necessary features of the concepts and conceptual relations by which
we think about reality
\end{description}

This view should be familiar from reading Descartes. Descartes also
thought that we could move from claims about dependence and distinctness
relations concerning our thoughts (e.g. that our thoughts of material
body depend, one and all, on our thought of extension) to conclusions
concerning the nature of the objects we think about in the
mind-independent world (e.g. that all the properties of a material
substance depend, as its modes, on extension). This assumption lies
behind Descartes's 'wax argument' concerning the nature of body and his
'real distinction' argument concerning the independence of mental
substance from the existence of any body. Descartes, of course, attempts
to justify his modal rationalism in the \emph{Meditations} by appealing to
the existence and veracity of God. Galileo, in contrast, provides no
such epistemological ground for his assumption.

\section{Descartes on Sensory Qualities}
\label{sec:orgf223981}
In the \emph{Sixth Meditation} Descartes argues that bodies exist because
otherwise God would be deceiving us regarding our idea of body. This
argument, if successful, tells us that bodies must at least have
qualities understood as modes of extension. What it does not do is tell
us whether bodies \emph{only} have qualities understood as modes of
extension. Descartes needs this stronger claim to hold because his
physics describes a material world that is uniform, differentiated into
shapes and sizes only by motion. He cannot allow that sensible qualities
are causally efficacious qualities distinct from the geometrical ones
specified by his arguments in the \emph{Fifth} and \emph{Sixth Meditations}.
Otherwise the uniformity of his picture of the physical world is
threatened.

Descartes's argument against sensory qualities is not obvious in any of
his writing. By the end of the \emph{Meditations} we seem to have a kind of
epistemic argument against sensory qualities.

\begin{enumerate}
\item We know that bodies must have geometrical qualities corresponding to
our clear and distinct geometrical ideas, otherwise God would be a
deceiver.
\item We do not know that bodies must have sensory qualities (e.g. color,
taste, smell, odor, etc.).
\item \(\therefore\) Until we have good reason, we should refrain from positing sensory
qualities in bodies.
\end{enumerate}

This is a kind of burden shifting argument against the proponent of
sensory qualities in bodies (see (\citeprocitem{7}{Rozemond 1998, 66–75}) for discussion).
Is there a more direct argument that Descartes makes?

It isn't clear, but in the \emph{Principles of Philosophy} he looks as if he
aims to provide the requisite argument that sensory qualities are not in
bodies in the same manner as geometric qualities.

\begin{quote-b}
For as regards hardness, our sensation tells us no more than that the
parts of a hard body resist the motion of our hands when they come
into contact with them. If, whenever our hands moved in a given
direction, all the bodies in that area were to move away at the same
speed as that of our approaching hands, we should never have any
sensation of hardness. And since it is quite unintelligible to suppose
that, if bodies did move away in this fashion, they would thereby lose
their bodily nature, it follows that this nature cannot consist in
hardness. By the same reasoning it can be shown that weight, colour,
and all other such qualities that are perceived by the senses as being
in corporeal matter, can be removed from it, while the matter itself
remains intact; it thus follows that its nature does not depend on any
of these qualities. (\textbf{Principles} II.4; AT 8A:42; CSMI:225)
\end{quote-b}

However, this argument leaves us wondering why an opponent couldn't
simply respond that Descartes hasn't shown that we can conceive of a
body that is not hard, but rather only that we can it conceive of a body
whose hardness we cannot experience. For this reason the argument
appears unsuccessful.

Given the burden shifting argument above, it is open to Descartes to
pursue a weaker, but still attractive (given his physics)
position---viz. one according to which we appeal to Ockham's razor, and
aim to construct a physical theory with as few ontological types as
possible (see (\citeprocitem{4}{Downing 2011, 113–14}) for discussion). Hence, if we can
get by in the course of offering physical explanations without positing
sensory qualities, then wouldn't this be grounds for thinking that
sensory qualities aren't really in bodies, or at least not in them in
the way that geometric qualities are?

In Pr I.70 Descartes mounts a more ambitious argument. He says,

\begin{quote-b}
it is quite different when we suppose that we perceive colours in
objects. Of course, we do not really know what it is that we are
calling a colour; and we cannot find any intelligible resemblance
between the colour which we suppose to be in objects and that which we
experience in our sensation. But this is something we do not take
account of; and, what is more, there are many other features, such as
size, shape and number which we clearly perceive to be actually or at
least possibly present in objects in a way exactly corresponding to
our sensory perception or understanding. And so we easily fall into
the error of judging that what is called colour in objects is
something exactly like the colour of which we have sensory awareness;
and we make the mistake of thinking that we clearly perceive what we
do not perceive at all. (AT 8A:34-5; CSMI:218)
\end{quote-b}

Descartes here argues that there is something \emph{unintelligible} about the
claim that colors (and presumably colors are no different from other
sensory qualities in this respect) resemble our sensations of them. If
it is simply unintelligible to attribute colors to bodies, then
Descartes has the stronger conclusion that he needs, for the claim that
bodies are colored (or hard, or have tastes, etc) would be an empty one.

But \emph{why} exactly is it that it is unintelligible for a body to be
colored in the manner in which we sense bodies as being colored?
Descartes doesn't make this clear, but we can speculate a bit based on
what he does say. For example, Descartes thinks that the essence of body
is extension, and that we are to understand all other properties of body
via this ``principle attribute'' or essence. So one reason for enforcing
the unintelligibility claim may be based on the fact that there is no
intelligible route from the property of being extended to the property
of being colored.

However (following (\citeprocitem{4}{Downing 2011})) there are at least two different
notions Descartes could be gesturing at with the intelligibility claim.
On the one hand, he could be arguing that color cannot be intelligibly
connected with extension, while on the other hand he could be arguing
that color cannot be derived from extension. Both of these arguments
construe the relation of a property (or mode) to a substance in
different ways. According to the first, in order for some property to be
considered a mode of a substance, it must be capable of being
intelligibly related to the substance's principle attribute. In the case
of color, however, just such a relation seems possible, since we cannot
conceive of color as being instantiated without it being instantiated in
some spatial region, and thus as having some extensive magnitude. So if
Descartes is attempting to exclude colors and other sensory qualities
from physical objects based on their compatibility with the essence of
body, this weak requirement won't do.

The stronger position is that color must be derivable from extension in
the same manner that, e.g., size and shape are derivable from extension.
It is difficult to see how this could be done. But it is also not clear
that Descartes properly motivates the stronger position. The fact that a
property needs to be conceivable through the essence of a substance does
not show that the property needs to \emph{derivable} from the essence. The
fact that Descartes never properly articulates why we should prefer the
stronger claim then ultimately undermines the success of his overall
argument.

\section{Locke on Ideas \& Qualities}
\label{sec:org64b94b6}
In the \emph{Essay} Locke, in a way similar to that of Galileo and Descartes,
distinguishes between our idea of a thing and the features or
'qualities' of the thing itself.

\begin{quote-b}
Whatever the mind perceives in itself---whatever is the immediate
object of perception, thought, or understanding---I call an idea; and
the power to produce an idea in our mind I call a quality of the thing
that has that power (E II.viii.8)
\end{quote-b}

So, for Locke, qualities are in objects, and those qualities are powers
to produce ideas in minds like ours. Locke thus seems to distinguish
between things that depend on \emph{us} -- ideas in minds, and things that
depend on \emph{objects} -- powers to cause those ideas. Moreover, Locke
distinguishes between the 'active' and 'passive' powers of an object.
Active powers are those powers of a substance to causally affect other
substances---fire has the active power to melt gold. Passive powers are
those powers of a substance \emph{to be causally affected by others} - gold
has the passive power to be melted by fire. It is tempting, and perhaps
correct, to read Locke here as using the term 'power' to describe the
modal features of an object - what it is possible (perhaps \emph{physically}
possible) for the object to affect and be affected by.

\subsection{Locke's Adaptation of the Primary/Secondary Distinction}
\label{sec:org09080fb}
Locke thinks that the kinds of qualities in objects may divided into two
classes - primary and secondary.

\begin{quote-b}
Qualities thus considered in bodies are, First, such as are utterly
inseparable from the body, in what state soever it be; and such as in
all the alterations and changes it suffers, all the force can be used
upon it, it constantly keeps; and such as sense constantly finds in
every particle of matter which has bulk enough to be perceived; and
the mind finds inseparable from every particle of matter, though less
than to make itself singly be perceived by our senses: v.g. Take a
grain of wheat, divide it into two parts; each part has still
solidity, extension, figure, and mobility: divide it again, and it
retains still the same qualities; and so divide it on, till the parts
become insensible; they must retain still each of them all those
qualities (II.viii.9).
\end{quote-b}

Primary qualities are, for Locke, the geometric/kinetic features of
substances -- their shape, size, motion, and solidity. These qualities
are in bodies independently of our perception of them, and cannot be
altered (at least as \emph{determinables}) or removed from bodies. Hence,
even in division of a body, its parts retain these determinable features
(though their determinate features may differ from the body of which
they were parts). Locke goes on.

\begin{quote-b}
Secondly, [there are qualities] which in truth are nothing in the
objects themselves but powers to produce various sensations in us by
their primary qualities, i.e. by the bulk, figure, texture, and motion
of their insensible parts, as colours, sounds, tastes, \&c. These I
call secondary qualities (II.viii.10, 14, 23).
\end{quote-b}

Next, there are the \emph{secondary} qualities, which are explanatorily less
significant, and are 'nothing in the objects themselves but powers'.
Hence there is some indication here that we have a contrast in the sense
in which primary as opposed to secondary qualities are \emph{real}. Primary
qualities are \emph{really in} bodies, while secondary qualities are \emph{mere
powers} to produce ideas in us. This way of putting the distinction
suggests something like Galileo's conclusion, that the secondary
qualities are merely subjective. But it also raises problems.

\subsubsection{Problems}
\label{sec:orgcb297b6}
As we saw above, Locke initially introduces the notion of a quality in
terms of a power to produce ideas. But his definition of a secondary
quality is in terms of its being 'nothing in the object itself' but a
power to produce ideas. But primary qualities are also qualities, and
thus powers to produce ideas. So where are we supposed to get a
\emph{distinction} between primary and secondary qualities? We seem to get
something like the following argument (see (\citeprocitem{6}{Rickless 1997}) for
discussion):

\begin{enumerate}
\item Shape is a primary quality
\item If shape is a primary quality then it is a quality
\item Qualities are powers to produce ideas
\item \(\therefore\) Shape is a power to produce an idea---presumably the idea of shape
\item But secondary qualities are also powers to produce ideas
\item \(\therefore\) There is no primary/secondary distinction between qualities
\end{enumerate}

This is certainly not the conclusion Locke wants us to draw. We appear
to have several options.

\begin{enumerate}
\item Deny that primary qualities are ``qualities'' in Locke's
\end{enumerate}
sense---i.e. deny that they are powers B. Agree that primary qualities
are powers but deny that they are the same sort of power as secondary
qualities C. Posit that Locke uses ``quality'' in two different senses:

\begin{itemize}
\item Quality\textsubscript{1}: X is a quality\textsubscript{1} iff X is a property
\item Quality\textsubscript{2}: X is a quality\textsubscript{2} iff X is a power to cause an idea in
one's mind

\item Maintain that there is only one sense to ``quality'' but distinguish
\end{itemize}
between real qualities and mere qualities

\begin{itemize}
\item ``Real'' qualities are qualities which are more than mere powers in
things, and exist even in the absence of things they affect
\item ``Mere'' qualities are no more than powers in things and cease to exist
when the things they affect are absent
\end{itemize}

There is a fair bit of textual evidence for (D).

\begin{quote-b}
The particular Bulk, Number, Figure, and Motion of the parts of Fire,
or Snow, are really in them, whether any ones Senses perceive them or
no: and therefore they may be called real Qualities, because they
really exist in those Bodies (II.viii.17)
\end{quote-b}

\begin{quote-b}
The Idea of Heat or Light, which we receive by our Eyes, or touch,
from the Sun, are commonly thought real Qualities, existing in the
Sun, and something more than mere Powers in it (II.viii.24)
\end{quote-b}

\begin{quote-b}
Were there no fit Organs to receive the impressions Fire makes on the
Sight and Touch; nor a Mind joined to those Organs to receive the
Ideas of Light and Heat, by those impressions from the Fire, or the
Sun, there would yet be no more Light, or Heat in the World, than
there would be Pain if there were no sensible Creature to feel it,
though the Sun should continue just as it is now, and Mount Aetna
flame higher than ever it did (II.xxxi.2)
\end{quote-b}

\begin{quote-b}
Let us consider the red and white colours in porphyry. Hinder light
from striking on it, and its colours vanish; when light returns, it
produces these appearances in us again. Can anyone think that any real
alterations are made in the porphyry by the presence or absence of
light; and that those ideas of whiteness and redness are really in
porphyry in the light, when it obviously has no colour in the dark?
(II.viii.19)
\end{quote-b}

In these four passages Locke seems to come extremely close to the
Galilean position that all secondary qualities are merely subjective,
while primary qualities are \emph{real}---that is, they really exist in
substances independently of the existence of any perceiver.

Another way of putting this point is that, according to Locke on
interpretation D, the real or primary qualities of an object are
\emph{intrinsic} to it. That is, they are had by it independent of any
relation that it might have to other things. ``Mere'' qualities, in
contrast, are \emph{relational}; they depend for their existence in part on
the existence and nature of something other than the objects whose
qualities they are. Locke indicates this in several places.

\begin{quote-b}
the greatest part of the ideas that make our complex idea of gold are
yellowness, great weight, ductility, fusibility\ldots{}all which ideas are
nothing else but so many relations to other substances; and are not
really in the gold, considered barely in itself, though they depend on
those real and primary qualities of its internal constitution
(II.xxiii.37)
\end{quote-b}

\begin{quote-b}
Put a piece of gold anywhere by itself, separate from the reach and
influence of all other bodies, it will immediately lose all its colour
and weight, and perhaps malleableness too (IV.vi.11)
\end{quote-b}

\subsection{Restating the Primary/Secondary Distinction}
\label{sec:orgb5fecc3}
With these distinctions in hand we might restate the distinction between
primary and secondary qualities as follows.

\begin{description}
\item[{Primary Qualities:}] Qualities of bodies which are ``really'' in them
in the sense of being both: (i) non-relational and (ii) of a
determinable which must be had by any body (e.g. solidity, extension,
size, shape)
\item[{Secondary Qualities:}] Qualities of bodies, which are ``merely'' in
them, and determined by their primary qualities to produce sensory
ideas in perceiving subjects. These qualities are both (i) relational
and (ii) of determinables which need not be had by every body
(e.g. colors, sounds, tastes)
\end{description}

\subsection{The Arguments for the Primary/Secondary Distinction}
\label{sec:org1efcd66}
Locke presents a number of different cases in which he thinks we clearly
see his primary/secondary distinction at work. Though it is not clear
whether he considers these cases to be \emph{arguments} for the distinction,
or merely illustrative of it, I'll be understanding them as arguments
(though perhaps very compressed ones). We'll take them in turn.\footnote{In my analysis of these three cases I follow (\citeprocitem{6}{Rickless 1997})
closely.}

\subsubsection{The Porphyry Case\footnote{\href{http://en.wikipedia.org/wiki/Porphyry\_(geology)}{Porphyry} is
an igneous rock of a reddish white color}}
\label{sec:orgce5c7e7}
\begin{quote-b}
Let us consider the red and white colours in Porphyre: Hinder light
but from striking on it, and its Colours Vanish; it no longer produces
any such Ideas in us: Upon the return of Light, it produces these
appearances on us again. Can any one think any real alterations are
made in the Porphyre, by the presence or absence of Light; and that
those Ideas of whiteness and redness, are really in Porphyre in the
light, when 'tis plain it has no colour in the dark? (II.viii.19)
\end{quote-b}

\begin{enumerate}
\item In the dark, porphyry is neither red nor white.
\item The real qualities possessed by porphyry in the dark are the same as
the real qualities possessed by porphyry in the light.
\item \(\therefore\) Redness and whiteness are not real qualities in porphyry, whether
in the light or in the dark.
\end{enumerate}

\subsubsection{The Almond Case}
\label{sec:orgb8478fe}
\begin{quote-b}
Pound an Almond, and the clear white Colour will be altered into a
dirty one, and the sweet Taste into an oily one. What real Alteration
can the beating of the Pestle make in any Body, but an Alteration of
the Texture of it?(II.viii.20)
\end{quote-b}

\begin{enumerate}
\item The pounding of an almond changes its color and taste.
\item The only real quality in an almond that it is possible to change by
pounding it is its texture.
\item Neither the color nor the taste of an almond is identical to its
texture.
\item \(\therefore\) The color and taste of an almond are not real qualities in it.
\end{enumerate}

\subsubsection{The Water Case}
\label{sec:org5dee04b}
\begin{quote-b}
Ideas being thus distinguished and understood, we may be able to give
an Account, how the same Water, at the same time, may produce the Idea
of Cold by one Hand, and of Heat by the other: Whereas it is
impossible, that the same Water, if those Ideas were really in it,
should at the same time be both Hot and Cold (II.viii.21)
\end{quote-b}

\begin{enumerate}
\item Water, at the same time, produces an idea of Heat when touched by one
hand and produces an idea of Coldness when touched by the other hand.
\item Water that produces an idea of Heat when touched by a hand has [the
quality] Heat.
\item Water that produces an idea of Coldness when touched by a hand has
[the quality] Coldness.
\item \(\therefore\) Water, at the same time, has Heat and Coldness. (From 1, 2, 3)
\item Heat and Coldness are opposites.
\item It is impossible for opposite real qualities to exist in the same
substance at the same time.
\item Heat is a real quality if and only if Coldness is a real quality.
\item \(\therefore\) Heat and Coldness are not real qualities. (From 4, 5, 6, 7)
\end{enumerate}

\section*{References}
\label{sec:orgc73138d}
\begin{hangparas}{1.5em}{1}
\hypertarget{citeproc_bib_item_1}{Blackwell, Richard J. 2008. \textit{Behind the Scenes at Galileo’s Trial: Including the First English Translation of Melchior Inchofer’s Tractatus Syllepticus}. University of Notre Dame Press.}

\hypertarget{citeproc_bib_item_2}{Blackwell, Richard J, and Paolo Antonio Foscarini. 1991. \textit{Galileo, Bellarmine, and the Bible: Including a Translation of Foscarini’s Letter on the Motion of the Earth}. South Bend, IL: University of Notre Dame Press.}

\hypertarget{citeproc_bib_item_3}{Downing, Lisa. 1998. “The Status of Mechanism in Locke’s Essay.” \textit{The Philosophical Review} 107 (3): 381–414.}

\hypertarget{citeproc_bib_item_4}{———. 2011. “Sensible Qualities and Material Bodies in Descartes and Boyle,” edited by Lawrence Nolan, 109–35. Oxford: Oxford University Press.}

\hypertarget{citeproc_bib_item_5}{Galilei, Galileo. 2008. \textit{The Essential Galileo}. Indianapolis, IN: Hackett Publishing.}

\hypertarget{citeproc_bib_item_6}{Rickless, Samuel C. 1997. “Locke on Primary and Secondary Qualities.” \textit{Pacific Philosophical Quarterly} 78 (3): 297–319.}

\hypertarget{citeproc_bib_item_7}{Rozemond, Marleen. 1998. \textit{Descartes’s Dualism}. Cambridge, MA: Harvard University Press.}
\end{hangparas}
\end{document}
