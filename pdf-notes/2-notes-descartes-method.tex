% Created 2022-01-20 Thu 12:27
% Intended LaTeX compiler: xelatex
\documentclass[12pt]{article}
                           
                           \input{/Users/roambot/.emacs.d/.local/custom-org-latex-classes/notes-setup-file.tex}
\author{PHIL 232}
\date{January 20, 2022}
\title{Descartes's Background \& Method}
\makeatletter
\newcommand{\citeprocitem}[2]{\hyper@linkstart{cite}{citeproc_bib_item_#1}#2\hyper@linkend}
\makeatother

\usepackage[notquote]{hanging}
\begin{document}

\maketitle
\tableofcontents

\setcounter{tocdepth}{6}
\tableofcontents


\section{Descartes's Background}
\label{sec:org59472d4}
\href{http://plato.stanford.edu/entries/descartes/}{René Descartes} (1596-1650) was born in La Haye in \href{http://en.wikipedia.org/wiki/Touraine}{the Touraine region} of France. His
father, Joachim, was part of the landed gentry and a parlamentarian in \href{http://en.wikipedia.org/wiki/Brittany}{Brittany}.
Descartes's mother, Jeanne Brochard, died when he was barely a year old. He was
raised primarily by his maternal grandmother. He had at least two older siblings, a
sister (Jeanne) and a brother (Pierre).

In 1606 or 1607, Descartes entered the newly founded \href{http://en.wikipedia.org/wiki/Jesuit}{Jesuit College} of La Fleche,
where he remained until 1614 or 1615. He recieved an education typical of this
school, with five to six years of 'grammar' school studying classical Latin, Greek,
and the major poets and orators (e.g. Cicero). This was followed by three years of
philosophical training, which was slightly less typical, since many students would,
after five years, go on to professional studies in law, medicine, or theology. The
Jesuit philosophical curriculum was based on the philosophy of Aristotle, and divided
into the then-standard topics of logic, morals, physics, and metaphysics. The Jesuits
also included mathematics in the final three years of study.

Descartes had mixed things to say about his time at La Fleche.

\begin{quote-b}
From my childhood I have been nourished upon letters, and because I was persuaded
that by their means one could acquire a clear and certain knowledge of all that is
useful in life, I was extremely eager to learn them. But as soon as I had completed
the course of study at the end of which one is normally admitted to the ranks of
the learned, I completely changed my opinion. For I found myself beset by so many
doubts and errors that I came to think I had gained nothing from my attempts to
become educated but increasing recognition of my ignorance. (\emph{Discourse} I; AT
VI:4-5)
\end{quote-b}

Descartes qualifies this by saying that ``I did not, however, cease to value the
exercises done in the Schools.'' (AT VI:5). As we learn in the \emph{Discourse}, Descartes
regarded the teachings at La Fleche as useful but uncertain, in large part due to
their lack of a secure method for acquiring knowledge. The articulation of such a
method is regarded by Descartes as one of his most significant contributions to
science (more on that below).

Descartes studied to be a lawyer, but then joined the army after getting his degree.
In the winter of 1619, while returning from Frankfurt to witness the coronation of
\href{http://en.wikipedia.org/wiki/Ferdinand\_II,\_Holy\_Roman\_Emperor}{Ferdinand II}, Descartes was stranded by a storm. During his time waiting out the
storm Descartes had a series of dreams the content of which determined him to set out
a new foundation for knowledge, and specifically, a new metaphysics, from which the
details of physics and a philosophy of nature more generally could be derived.

Eventually, Descartes moved to the Netherlands (1629) and settled there. He had one
daughter, by a servant (Helene), whom he named Francine. Unfortunately the daughter
died young. Descartes is rumored to have carried a doll made in her size and likeness
with him on his travels. By 1648 he had attained some fame on the European continent,
and was invited by Queen Christina of Sweden to be her tutor. He accepted and moved
to Sweden in 1649. There he was required to give her instruction in philosophy at
five in the morning, for five hours at a time, three days a week. It is speculated
that the early morning regimen, to which Descartes was unaccustomed, along with the
cold weather, contributed to his contracting a respiratory infection that led to his
death on the 11\textsuperscript{th} of February, 1650. He was fifty-three years old.

\section{Philosophical Work}
\label{sec:org17caa75}
For a general overview of Descartes's work see the two entries in the Stanford
Encyclopedia of Philosophy \href{http://plato.stanford.edu/entries/descartes/}{here} and \href{http://plato.stanford.edu/entries/descartes-works/}{here}.

Descartes made significant contributions in mathematics (e.g. the algebraic method
used to develop \href{http://en.wikipedia.org/wiki/Analytic\_geometry}{analytic geometry}; see (\citeprocitem{1}{Boyer and Merzbach 2011, 309–12})), in optics, and in
physics. We shall discuss two major works of his, viz., The \emph{Discourse on the Method
of Rightly Conducting One's Reason and Seeking Truth in the Sciences}, and the
\emph{Meditations on First Philosophy}. Below I give a very general overview of the
\emph{Discourse} and the \emph{Meditations}.

\subsection{The \emph{Discourse on the Method} (1637)}
\label{sec:orgdef538d}
The \emph{Discourse on the Method of Rightly Conducting One's Reason and Seeking Truth in
the Sciences} was written over a period of several years and published (in French) in
1637, when Descartes was 41. It combined discussion of scientific methodology with a
series of arguments, in accordance with that offered methodology, all aimed at
articulating a metaphysical foundation for physics.

The \emph{Discourse} is Descartes's first published work, and comes approximately four years
after he had abandoned work on a scientific treatise he titled \emph{The World}, largely due
to worries concerning possible religious persecution for the doctrines espoused
within. I discuss \emph{The World} in a bit more detail below. In the end Descartes decided
to try and repackage many of his arguments from \emph{The World} without some of the more
controversial natural scientific material, and thus we have the \emph{Discourse}.

The \emph{Discourse} provides background concerning Descartes's early education and his
motivation for articulating a precise method for acquiring knowledge. The \emph{Discourse}
also contains parts of the previous and unfinished \emph{World}, including a work on optics
(the \emph{Dioptrics}), and works on meteorology and geometry.

The \emph{Discourse} consists of six parts. Part I of the \emph{Discourse} is autobiographical,
setting out how Descartes came to think that there was a need for a new method in
gaining knowledge. In Part II, Descartes suggests that the surest way to knowledge is
via a radical reconstruction of one's beliefs, allowing to remain only those of which
one was certain.

\begin{quote-b}
regarding the opinions to which I had hitherto given credence, I thought that I
could not do better than undertake to get rid of them, all at one go, in order to
replace them afterwards with better ones, or with the same ones once I had squared
them with the standards of reason. I firmly believed that in this way I would
succeed in conducting my life much better than if I built only upon old foundations
and relied only upon principles that I had accepted in my youth without ever
examining whether they were true. (\emph{Discourse} II; AT VI:13-14)
\end{quote-b}

Descartes was partly motivated in this radical reconstruction by his
mathematical discoveries. In particular, he had become convinced by his
aforementioned mathematical discoveries concerning the algebraic representation
and solution of both geometric and arithmetical problems that the barriers to
knowledge were largely methodological in nature. This led to his being convinced
of the existence of an underlying discipline of which geometry, arithmetic,
astronomy, and the theory of harmony were all expressions (\citeprocitem{4}{Gaukroger 2006, 59–60}). He called this discipline \emph{mathesis universalis} or ``universal
mathematics.'' Ultimately, Descartes believed that there was an even more
fundamental basis for the method used by his universal mathematics. This is the
method which he first articulates in the unpublished \emph{Rules for the Direction of
our Native Intelligence} (1628) and then presents in a more condensed fashion in
part two of the \emph{Discourse}. Here Descartes sets out four rules for ordering one's
thoughts and acquiring knowledge:

\begin{quote-b}
The first was never to accept anything as true if I did not have evident knowledge
of its truth: that is, carefully to avoid precipitate conclusions and
preconceptions, and to include nothing more in my judgements than what presented
itself to my mind so clearly and so distinctly that I had no occasion to doubt it.
\end{quote-b}

\begin{quote-b}
The second, to divide each of the difficulties I examined into as many parts as
possible and as may be required in order to resolve them better.
\end{quote-b}

\begin{quote-b}
The third, to direct my thoughts in an orderly manner, by beginning with the
simplest and most easily known objects in order to ascend little by little, step by
step, to knowledge of the most complex, and by supposing some order even among
objects that have no natural order of precedence.
\end{quote-b}

\begin{quote-b}
And the last, throughout to make enumerations so complete, and reviews so
comprehensive, that I could be sure of leaving nothing out. (AT VI:18-19)
\end{quote-b}

It is this method that Descartes follows in the \emph{Meditations} in order that he may
provide a proper foundation for natural science.

Part III of the \emph{Discourse} discusses practical and moral considerations of adhering to
these rules. Part IV offers arguments for the existence of God and the soul, which
form the basis of his metaphysical foundation for physics. These arguments will be
revisited in the \emph{Meditations}. Part V lays out some aspects of Descartes's theory of
matter, his physics, and some discussion of his biological theory. Finally, in Part
VI, Descartes discusses aspects of his views and their relation to the recently
censured views of \href{https://plato.stanford.edu/entries/galileo/}{Galileo}, as well as his reasons for publishing the \emph{Discourse} in the
manner that he did.

The Appendices of the \emph{Discourse} are from Descartes's unpublished work, ``\emph{The World}''
(1633), which he wrote in the hopes of providing a unified physical theory of nature
(the human being, the Earth, and the heavens). \href{https://philosophy.sas.upenn.edu/people/gary-hatfield}{Gary Hatfield} (a historian of
philosophy) describes the significance of the work thusly,

\begin{quote-b}
[ In \emph{The World} ] Descartes went far beyond the Copernican hypothesis that our
Sun lies at the center of the universe with the Earth moving about it. He
contended that the Earth is one among many planets, revolving around many
different suns distributed throughout the cosmos. He further proposed that the
whole universe is made of one kind of matter, which follows one set of laws.
He invented the concept of a single universe, filled with matter having a few
describable properties and governed by a few laws of motion. While others,
including ancient [Greek] atomists and Stoics, had sketched part of this new
picture, Descartes' vision of a unified physics governed by a few laws of
motion was far richer and more detailed. Its combination of breadth and unity
was unprecedented in his earlier work with Beeckman, or in the works of
Copernicus, Galileo, or Kepler. This unified vision set the framework for
Newton's subsequent unification of mechanics and astronomy.
(\citeprocitem{5}{Hatfield 2003, 18})
\end{quote-b}

When Descartes was nearing completion of \emph{The World} in 1633, he learned of the
Catholic church's condemnation of Galileo for defending the Copernican hypothesis
that the Earth orbited the sun. Not wanting to be branded a criminal by the church
Descartes decided not to publish the work and contemplated burning all of his papers.
\emph{The World} was eventually published after Descartes's death in 1664.

\subsection{The \emph{Meditations on First Philosophy} (1641)}
\label{sec:org38f2f6f}
In 1647 Descartes published a French version of his Latin work \emph{Principles of
Philosophy} (1644), in which he laid out his philosophical system. In the preface to
this translation he famously compares the structure of his theory to a tree, whose
different parts represent the different sciences.

\begin{quote-b}
The first part of philosophy is metaphysics, which contains the principles of
knowledge, including the explanation of the principal attributes of God, the
non-material nature of our souls and all the clear and distinct notions which are
in us. The second part is physics, where, after discovering the true principles of
material things, we examine the general composition of the entire universe and
then, in particular, the nature of this earth and all the bodies which are most
commonly found upon it, such as air, water, fire, magnetic ore and other minerals.
Next we need to examine individually the nature of plants, of animals and, above
all, of man, so that we may be capable later on of discovering the other sciences
which are beneficial to man. Thus the whole of philosophy is like a tree. The roots
are metaphysics, the trunk is physics, and the branches emerging from the trunk are
all the other sciences, which may be reduced to three principal ones, namely
medicine, mechanics and morals. By `morals' I understand the highest and most
perfect moral system, which presupposes a complete knowledge of the other sciences
and is the ultimate level of wisdom. (AT 9B:14 in (\citeprocitem{2}{Descartes 1985, 186}))
\end{quote-b}

The \emph{Meditations}, published in Latin just prior to the \emph{Principles}, in 1641, was
Descartes's attempt to argue in a systematic way for the foundations of his physics,
and thus, given the above metaphor of science as a tree, the ``roots'' of all human
knowledge. Descartes also wanted to present his views in a manner that he hoped would
be congenial to 
(he hoped for, but never received, the endorsement of the Jesuits). It is, by and
large, an elaboration of the arguments presented in part four of the \emph{Discourse},
which preceded his discussion in part five (and the appendices) of the nature of the
natural world. It thus articulates, at least according to Descartes, all the
principles required for the metaphysical basis (the roots of the tree) of his
physics, and the other natural sciences more generally.

However, Descartes himself does not present the argument of the \emph{Meditations} in this
way. Instead, Descartes highlights, in the subtitle to the work, the proofs it
contains of God's existence, and of the distinctness of soul from body. In his letter
to the church faculty of the Sorbonne, Descartes even goes so far as to suggest that
the primary aim of the work is to help convert ``unbelievers'' (i.e. atheists) (AT
7:1-2). However, to Marin Mersenne, one of his confidants in Paris, Descartes writes:

\begin{quote-b}
I will say to you, just between us, that these six Meditations contain all the
foundations of my Physics. But, please, you must not say so; for those who favor
Aristotle would perhaps have more difficulty in approving them; and I hope that
those who will read them will unwittingly become accustomed to my principles, and
will recognize the truth, before they notice that my principles destroy those of
Aristotle. (AT 3:297--8 in (\citeprocitem{3}{Descartes 1991, 173}))
\end{quote-b}

So, despite hoping to win favor with various schools of scholastic Aristotelianism,
and the Jesuits in particular, Descartes nevertheless took himself to be articulating a
theory diametrically opposed to Aristotle's (and the medieval scholastic versions
thereof).

The \emph{Meditations} consists of six parts, each of which takes up a different but related
topic. In order, they are (1) the aims of the work and the method of doubt; (2) the
\emph{cogito} and our knowledge of body; (3) proof of God's existence; (4) judgment and
error; (5) our knowledge of essence and the ontological proof of God's existence; (6)
the real distinction between mind and body and knowledge of the external (i.e.
material) world.

In the course of discussing these topics, Descartes argues that matter has essential
features wholly different from those put forth by Aristotlelian philosophy. He will
also argue that the mind and body are wholly distinct and different substances. He
thus fundamentally denies the hylomorphism used by the medievals to argue that the
human being as well as other material bodies are composites of matter and substantial
form.

\subsubsection{Method \& the Meditations}
\label{sec:orgcdd60f0}
Recall that Descartes's method has four parts: (1) accepting as true only that for
which one has proper evidence---viz., clear and vivid ideas; (2) resolving problems
into their simplest parts; (3) moving from the simple and known to the complex; and
(4) thoroughly reviewing and checking one's work to be sure it is comprehensive and
complete. Perhaps unsurprisingly, given that Descartes was a mathematician, this
method is well suited to mathematical practice. But it was one of Descartes's hopes
that he might resolve many of the disputes to which metaphysics was prone by bringing
the methods of mathematics to it.

The \emph{Meditations} carries out Descartes's method in a fairly straightforward manner.
First, (as is clear from \emph{Meditation I}) Descartes follows a ``method of doubt'' so to
accept only what is vividly and clearly perceived to be true. Second, he attempts to
resolve the fundamental issues of metaphysics into the simplest parts possible, which
he then (third) pursues in increasing order of complexity. Finally (fourth), in
reviewing his whole argument, by the end of the \emph{Meditations} he purports to have given
us an argument that provides the complete basic principles of philosophy.

\subsubsection{The Meditation as Literary Form}
\label{sec:org5648990}
It is important to note that, despite Descartes's insistence on bringing mathematical
methods to bear on issues of fundamental philosophy, he does not present the work in
mathematical, or ``\href{https://iep.utm.edu/geo-meth/}{geometric}'', form. Instead, he presents his arguments in a series of
``meditations''. The meditation as literary form was a \href{http://plato.stanford.edu/entries/medieval-literary/\#MedSol}{common trope of medieval
philosophy and theology}. Typically, the style is introspective, and often as a dialog
with oneself or \href{http://plato.stanford.edu/entries/augustine/\#ReaCon}{with God}. There is supposed to be personal and/or spiritual growth as
the meditations progress. This also goes along with the important first-person aspect
of Descartes's \emph{Meditations}, which raises the question of who the `I' refers to in the
\emph{Meditations} proper. Is it Descartes? This seems unlikely, as the winter storm in
which he is trapped in the small cottage by the stove occurs in 1619, at least nine
years before he develops his metaphysical system. What's more, Descartes clearly does
not himself think some of the things which the Meditator does, such as that the
senses are the primary basis of all knowledge. So perhaps the Meditator is supposed
to be someone else?---perhaps the reader?

Whatever the answers to these questions, it seems clear that the Meditator is
supposed to take an intellectual journey in the course of proceeding through each
section of the work so that, in the end, he or she comes to a proper understanding of
the truth---viz. the truth of the Cartesian metaphysical system.

\section*{References}
\label{sec:org47bfa37}
\begin{hangparas}{1.5em}{1}
\hypertarget{citeproc_bib_item_1}{Boyer, Carl B, and Uta C Merzbach. 2011. \textit{A History of Mathematics}. Hoboken, NJ: Wiley.}

\hypertarget{citeproc_bib_item_2}{Descartes, René. 1985. \textit{The Philosophical Writings of Descartes}. Cambridge: Cambridge University Press.}

\hypertarget{citeproc_bib_item_3}{———. 1991. \textit{The Philosophical Writings of Descartes: The Correspondence}. Cambridge: Cambridge University Press.}

\hypertarget{citeproc_bib_item_4}{Gaukroger, Stephen. 2006. “Knowledge, Evidence, and Method.” In \textit{The Cambridge Companion to Early Modern Philosophy}, edited by Donald Rutherford, 39–66. Cambridge: Cambridge University Press.}

\hypertarget{citeproc_bib_item_5}{Hatfield, Gary. 2003. \textit{Routledge Philosophy Guidebook to Descartes and the Meditations}. London: Routledge.}
\end{hangparas}
\end{document}
