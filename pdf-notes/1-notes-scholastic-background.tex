% Created 2022-01-20 Thu 11:09
% Intended LaTeX compiler: xelatex
\documentclass[12pt]{article}
                           
                           \input{/Users/roambot/.emacs.d/.local/custom-org-latex-classes/notes-setup-file.tex}
\author{PHIL 232}
\date{January 18, 2022}
\title{The Scholastic Aristotelian Background}
\makeatletter
\newcommand{\citeprocitem}[2]{\hyper@linkstart{cite}{citeproc_bib_item_#1}#2\hyper@linkend}
\makeatother

\usepackage[notquote]{hanging}
\begin{document}

\maketitle
\setcounter{tocdepth}{6}
\tableofcontents


The philosophical doctrines to which the moderns were primarily reacting were those
of the 'School' philosophies as advocated by teachers or ``\href{http://en.wikipedia.org/wiki/Scholasticism}{scholastics}'' at the major
univerisities of the medieval era (Oxford, Cambridge, Paris, Bologna, Salamanca).
These centers of learning trained for the main professions (medicine, law, and
theology) and ``natural philosophy'' via a ``liberal arts'' curriculum of grammer,
rhetoric, and logic, (called the ``trivium'') and mathematics, geometry, astronomy, and
music (called the ``quadrivium''). ``Scholastic philosophy'', in this broad sense, thus
constituted the primary set of philosophical doctrines in the \href{http://plato.stanford.edu/entries/medieval-philosophy/}{medieval period}.

Scholastic philosophy cannot be easily characterized. Philosophers from the
scholastic period occupied an astonishing array of positions regarding a variety of
issues in \href{http://plato.stanford.edu/entries/skepticism-medieval/}{epistemology}, \href{http://plato.stanford.edu/entries/medieval-syllogism/}{logic}, \href{http://plato.stanford.edu/entries/medieval-terms/}{philosophy of language}, \href{http://plato.stanford.edu/entries/representation-medieval/}{philosophy of mind}, and
\href{http://plato.stanford.edu/entries/universals-medieval/}{metaphysics}. However, there were several general doctrines or methods of inquiry that
were broadly shared by the scholastics, and were gradually modified, questioned, and
ultimately rejected, by modern philosophical theory. These doctrines or methods often
have roots in the philosophy of \href{http://plato.stanford.edu/entries/aristotle/}{Aristotle}, whose influence in Medieval philosophy was
so pervasive that he was commonly referred to simply as ``the Philosopher'', and the
standard manner for writing philosophical works was to present them as commentaries
on Aristotle's writings.

The period of scholastic philosophy which would have had the greatest influence on
the Modern period roughly follows that of the period from \href{http://plato.stanford.edu/entries/augustine/}{Augustine} (354--430), and
the synthesis of Christian and classical Greek---especially
\href{https://plato.stanford.edu/entries/neoplatonism/}{neoplatonist}---philosophy, to \href{http://plato.stanford.edu/entries/aquinas/}{Aquinas} (1225--1274), who combined Catholic doctrine
with Aristotelian metaphysics in original and influential ways. Subsequent to
Aquinas's death, scholastic doctrine would develop considerably, see criticism from
both within and without, and ultimately face a radical and precipitous decline in
influence, only to have its fortunes briefly revived in the sixteenth and seventeenth
centuries in the work of figures like \href{https://plato.stanford.edu/entries/suarez/}{Francisco Suárez} (1548--1617).

Below I discuss five central ways in which Scholastic, and ultimately early modern,
philosophy was influenced by Aristotle. These are his empiricism, his ``hylomorphism'',
his conception of causation and explanation, his view of science, and his
cosmology.\footnote{It's important to note that many scholastics reject one or more
of these Aristotelian doctrines, and many endorsed them in ways
that allow for only very general agreement with what Aristotle
plausibly held. Nevertheless, the doctrines discussed below form
a backdrop against which many modern philosophical views develop.}

\section{Aristotelian Empiricism}
\label{sec:org6a06f2b}
One of the central doctrines of Aristotelianism is a form of what we now call
'empiricism.' Represented in the scholastic era via the motto '\emph{nihil est in
intellectu quod prius non fuit in sensu}' (there is nothing in the intellect that was
not first in the senses), this doctrine holds that the origin of all thinkable
content must be traced to our sensory interaction with the world. There is nothing of
which we can think that is not first given, in some form or another, through the
senses. This entails that all knowledge must ultimately be grounded in (i.e. depend
on) sensory experience. This in turn require two things. First, reality, if we are to
know it in the way demanded by proper science (more on this below), must be
accessible to the senses. Second, our sensory faculties (vision, touch, smell, etc.)
must be reliable in bringing us into cognitive contact with reality. If our sensory
apprehension of reality systematically distorts its character, or otherwise causes us
to have false beliefs concerning reality, then we cannot rely on the senses in the
acquisition of knowledge. The trio of questions concerning whether the content of our
concepts must come from experience, whether knowledge depends on our having
experience, and whether the senses are a reliable means of apprehending reality,
would be central to the philosophical systems of modern thinkers such as Descartes,
Locke, Leibniz, Hume and Kant.

\section{The Four Causes}
\label{sec:orge4598d2}
Aristotle believed that in order to be properly said to ``know'' something one must
understand why it is the case. In connection with this conception of knowledge in
terms of an understanding of the ``why'' of things, Aristotle argues that we must know
how a thing is caused to be (APost. 71 b 9--11; see also APost. 94 a 20). In service
of this he articulates an explanatory scheme involving four distinct notions of
``cause'' (\emph{Aitia} [Greek: αἰτία]). He puts it this way in his \emph{Physics}:

\begin{quote-b}
One way in which cause is spoken of is that out of which a thing comes to be and
which persists, e.g. the bronze of the statue, the silver of the bowl, and the
genera of which the bronze and the silver are species.

In another way cause is spoken of as the form or the pattern, i.e. what is
mentioned in the account (\emph{logos}) belonging to the essence and its genera, e.g. the
cause of an octave is a ratio of 2:1, or number more generally, as well as the
parts mentioned in the account (\emph{logos}).

Further, the primary source of the change and rest is spoken of as a cause,
e.g. the man who deliberated is a cause, the father is the cause of the child, and
generally the maker is the cause of what is made and what brings about change is a
cause of what is changed.

Further, the end (\emph{telos}) is spoken of as a cause. This is that for the sake of
which (\emph{hou} \emph{heneka}) a thing is done, e.g. health is the cause of walking about. 'Why
is he walking about?' We say: 'To be healthy'--- and, having said that, we think we
have indicated the cause. (Phys. 194b23--35)
\end{quote-b}

This results in four distinct kinds of explanation, which answer four
questions---viz., \emph{what is it?} (formal), \emph{what is it made of?} (material), \emph{what
brought it about?} (efficient), and \emph{what is it for?} (final). We can categorize
these as follows:

\begin{enumerate}
\item The \textbf{material cause}: that from which something is generated and out of which it is
made, e.g. the bronze of a statue.
\item The \textbf{formal cause}: the structure which the matter realizes and in terms of which it
comes to be something determinate, e.g., the shape of the president, in virtue of
which this quantity of bronze is said to be a statue of a president.
\item The \textbf{efficient cause}: the agent primarily responsible for a quantity of matter's
coming to be informed, e.g. the sculptor (or the sculptor's knowledge of art) who
shaped the quantity of bronze into its current shape, the shape of the president.
\item The \textbf{final cause}: the purpose or goal of the compound of form and matter, e.g. the
statue was created for the purpose of honoring the president.
\end{enumerate}

According to Aristotle, citation of at least some of the four causes is both
\emph{necessary} and \emph{sufficient} for providing an adequate explanation. As \href{http://plato.stanford.edu/entries/aristotle/\#FouCauAccExpAde}{Chris Shields}
points out, since not all four causes are present in every instance to be explained,
it cannot be that \emph{all four} are necessary for an adequate explanation.

\begin{quote-b}
for example, coincidences lack final causes, since they do not occur for the sake
of anything; that is, after all, what makes them coincidences. If a debtor is on
his way to the market to buy milk and she runs into her creditor, who is on his way
to the same market to buy bread, then she may agree to pay the money owed
immediately. Although resulting in a wanted outcome, their meeting was not for the
sake of settling the debt; nor indeed was it for the sake of anything at all. It
was a simple co-incidence. Hence, it lacks a final cause. (\citeprocitem{6}{Shields 2015})
\end{quote-b}

One of the central disputes in the modern era concerns whether appeal to anything
more than material and efficient causal explanations are needed in an adequate
science of nature. A characteristic position taken by Descartes, Locke, Spinoza, and
others was to deny that formal or final causes were explantorily necessary or useful.
In the case of formal causes, many became suspicious of Aristotle's hylopmorphism,
and thus his notion of a ``form'' as opposed to its ``matter''. With respect to final
causes, many in the modern era argued that there are no final causes other than those
explicitly dependent on the will of human beings (this latter position is especially
clear in Spinoza and, in slightly weaker form, Kant).

\section{Hylomorphism}
\label{sec:orgfcd7bdc}
Aristotle's ``\href{http://plato.stanford.edu/entries/aristotle/\#Hyl}{hylomorphism}'' is another central part of his philosophy, and one of the
most influential parts of his overall doctrine. Its basic conceptual vocabulary is
retained not only in the medieval period, but also well into the modern period.
``Hylomorphism'' is a compound word composed of the Greek terms for matter (\emph{hulê}) and
form or shape (\emph{morphê}). While the Greek term has been retained to describe
Aristotle's view (and other views like it), we could also simply refer to the
doctrine in English as ``matter-formism.''

The doctrine of matter and form, which is the basis of hylomorphism, is perhaps best
illustrated by examples. Consider two types of change. First, there is \emph{generative}
change, as in the building of a house where none existed before, or the conception of
a child. Second, there is \emph{qualitative} change, wherein something which already exists
is altered, as when a piece of fruit changes color, or a person gets older. Aristotle
endorsed a principle, popular in both his time and in the modern era (and in
\href{http://www.nytimes.com/2012/03/25/books/review/a-universe-from-nothing-by-lawrence-m-krauss.html}{contemporary physics} as well) that nothing can come from nothing. So he needed to
explain both generative and qualitative change in a way that didn't violate this
principle. This meant explaining change in terms of (i) something underlying and
persisting; and (ii) something gained or lost.\footnote{See (\citeprocitem{5}{Shields 2007, chap. 2}) for helpful discussion.} These two explanations of change
correspond to Aristotle's notions of \emph{matter} and \emph{form} respectively. Matter is what
persists through all change, like the bricks of a house, which can persist
independently of whether or not the house itself does. And form is what may be gained or
lost, either in alteration (losing the form \emph{young} and gaining the form \emph{old}) or in
generation (losing the form \emph{lump} and gaining the form \emph{statue}).

The notions of matter and form are paired with two other notions, that of
potentiality and actuality. Matter is something that potentially has some feature F,
as a lump of bronze may potentially take the shape of a statue. Form is that which
makes some bit of matter, which is potentially F, \emph{actually} F (literally, in the case
of the statue, less literally in the case of, e.g., a human being).\footnote{See \href{http://plato.stanford.edu/entries/aristotle/\#Hyl}{Shields' discussion} and the related SEP entry on \href{https://plato.stanford.edu/entries/aristotle-causality/}{Aristotle's theory of
causation}, for further explanation and references.}

The basic unit of explanation for Aristotle is called a \href{http://plato.stanford.edu/entries/aristotle/\#Sub}{\emph{substance}}. Substances, for
Aristotle as well as many of the Scholastics who followed him, possess a particular
way of being or existing. Substances exist in themselves, and not in any other thing.
Substances thus \textbf{substand}, and they also \textbf{subsist}, or exist independently of other
things. Because of these two features, substances are loci of explanation. Properties
inhere \textbf{in} (have their being in) the substances which posses them, but not the other
way around.

Non-substantial beings, such as properties or events, might stand in as subjects, and
thus as loci of explanation, in limited cases. For example, one might say of a fight
that it is vicious, or of a rainbow that it is beautiful. Similarly, one might hope
of justice that it is blind. But the fight and the rainbow are adjectival on the
beings which constitute them (respectively, the fighters and the raindrops), while
talk of the blindness of justice, if we are to avoid reifying the property, is purely
metaphorical. But these relative substances are not ultimately explanatorily primary
because their being, and thus the explanation of their features, depends ultimately
on the substances that constitute them.

The conception of a substance as a kind of fundamental being connects with the issue
of hylomorphism and change in the following way. Since all the substances in nature,
including human beings, exhibit change, all substances are complexes of matter and
form. The form possessed by some parcel of matter determines it as the kind of thing
that it is, e.g. determines a bit of matter as \emph{gold} or as \emph{living}. Moreover, the kind
of form possessed by a particular substance that makes it the kind of thing that it
is thereby constrains the sorts of properties that can inhere in the substance. Thus,
according to the Aristotelian picture, the changing patterns of properties that we
find instantiated in nature are due to various kinds of form/matter complexes, the
most fundamental of which are substances, whose way of existing insures that they are
the fundamental loci of explanation. Substances, their features, and their behavior,
are therefore the subject matter of any ``proper'' science, so we should look now at the
structure of Aristotelian science.

\section{Demonstrative Science}
\label{sec:org8aa631a}
We've seen that the Aristotelian worldview conceives of knowledge as ultimately
dependent on the senses, and as presenting a world structured in terms of form
and matter. Proper knowledge of the world also requires understanding it in
terms of a four-part explanatory causal framework. These doctrines are the
backdrop to the Aristotelian conception of a science of the natural world---what
we now think of as ``science'' proper. However, our contemporary notion of
scientific knowledge, and ``scientists'' as the people who practice science, only
came into existence in the \href{https://en.wikipedia.org/wiki/Scientist\#Historical\_development\_and\_etymology\_of\_the\_term}{19th century}. Before then, and going back to
\href{http://plato.stanford.edu/entries/aristotle-natphil/}{Aristotle's natural philosophy}, the dominant conception of theoretical
knowledge, known in Latin as \emph{`Scientia'} (in Greek as \emph{`Epistêmê'} (ἐπιστήμη)),
concerned any body of knowledge organized according to some principle or set of
principles that exemplify the explanatory relations of what is best known and
explanatorily basic as the basis of what is least known and explanatorily
derivative (see also (\citeprocitem{2}{Jardine 1988}; \citeprocitem{4}{Randall 1961}; \citeprocitem{1}{De Jong and Betti 2010})). A
'science' was thus distinguished from a mere aggregate of known facts in virtue
of the presence of such explanatory connections between facts.

For example, when asked why trees lose their leaves in the fall, one might
reply, ``because the wind blows them off.'' One might even label trees which
feature this characteristic as `\href{http://en.wikipedia.org/wiki/Deciduous}{deciduous}'. Neither the label, in and of itself,
nor the description would be a particularity deep or fruitful form of
explanation. A better one would be one that articulated \emph{why} there is this
connection between season and plant behavior. For example, that diminished
sunlight in the autumn inhibits the production of chlorophyll, which is required
for photosynthesis, and without photosynthesis trees go dormant and shed their
leaves. Now the ``deciduous'' label has a more explanatory role. We can deduce,
from the fact that a tree is deciduous, that it has certain characteristics, and
that these characteristics play an important explanatory role in understanding
the tree's behavior. Not only that, but the explanation is also importantly
\emph{asymmetrical} in nature. A tree is deciduous in virtue of its failure to produce
chlorophyll at particular times, and in turn, this lack of chlorophyll
production explains why the tree fails to photosynthesize, rather than the other
way around.

Aristotle puts the notion of scientific explanation---knowledge why
rather than mere knowledge that---this way:

\begin{quote-b}
We suppose ourselves to possess unqualified scientific knowledge of a thing, as
opposed to knowing it in the accidental way in which the sophist knows, when we
think that we know the cause on which the fact depends, as the cause of that fact
and of no other, and, further, that the fact could not be other than it is\ldots{}.The
proper object of unqualified scientific knowledge is something which cannot be
other than it is. (Aristotle, Posterior Analytics, I.2)
\end{quote-b}

Note that Aristotle believed that scientific explanation not only captures
asymmetrical explanatory relations but also ones which are, in some sense, \emph{necessary}.
One way in which this explanatory relationship might be modeled is via \href{http://plato.stanford.edu/entries/aristotle-logic/}{syllogism}. A
syllogism is an argument in which a judgment---the ``conclusion''---logically follows
from some other set of judgments---the ``premises'' or ``assumptions''. In Aristotle's
logic (and in the logic used from the Medieval and into the Early Modern era) the
premises were typically understood to be judgments of what was already known. So a
conclusion C follows from some premises A and B, if and only if it is impossible for
C to be false while A and B are true (and known). Hence, Aristotle takes proper
scientific explanation to be structured in the manner of a logical derivation, where,
from basic knowledge, one derives other knowledge via logical argument. As Chris
Shields \href{http://plato.stanford.edu/entries/aristotle/\#Sci}{puts it,}

\begin{quote-b}
the currency of science is demonstration ([from the Greek] ``\emph{apodeixis}''), where a
demonstration is a deduction with premises revealing the causal structures of the
world, set forth so as to capture what is necessary and to reveal what is better
known and more intelligible by nature (APo 71b33--72a5, Phys. 184a16--23, EN
1095b2--4).
\end{quote-b}

Hence, the structure of scientific knowledge is \emph{demonstrative} and its demonstrations
exhibit asymmetric explanatory relations between the things known. The demonstrative
structure of scientific knowledge raises the problem of how such demonstrations get
started. How do we know the premises of a demonstrative argument? As Aristotle puts
it,

\begin{quote-b}
Some people think that since knowledge obtained via demonstration requires the
knowledge of primary things, there is no knowledge. Others think that there is
knowledge and that all knowledge is demonstrable. Neither of these views is either
true or necessary. The first group, those supposing that there is no knowledge at
all, contend that we are confronted with an infinite regress. They contend that we
cannot know posterior things because of prior things if none of the prior things is
primary. Here what they contend is correct: it is indeed impossible to traverse an
infinite series. Yet, they maintain, if the regress comes to a halt, and there are
first principles, they will be unknowable, since surely there will be no
demonstration of first principles---given, as they maintain, that only what is
demonstrated can be known. But if it is not possible to know the primary things,
then neither can we know without qualification or in any proper way the things
derived from them. Rather, we can know them instead only on the basis of a
hypothesis, to wit, \emph{if} the primary things obtain, then so too do the things derived
from them. The other group agrees that knowledge results only from demonstration,
but believes that nothing stands in the way of demonstration, since they admit
circular and reciprocal demonstration as possible. (\emph{APo}. 72b5--21)
\end{quote-b}

Aristotle presents here a dispute between, on the one hand, those who deny that there
is any knowledge, because knowledge depends on demonstration, and there is an
infinite regress of demonstrated knowledge that we cannot complete. On the other hand
there are those that allow for the possibility of non-demonstrative knowledge, but
deny that we can have such knowledge. Aristotle, in contrast to these two positions,
contends that if all scientific knowledge is derived from things already known there
has to be at least some knowledge that is basic and non-demonstrative. Otherwise our
claims to justification would run in a circle and nothing would be known.

\begin{quote-b}
We contend that not all knowledge is demonstrative: knowledge of the immediate
premises is indemonstrable. Indeed, the necessity here is apparent; for if it is
necessary to know the prior things, that is, those things from which the
demonstration is derived, and if eventually the regress comes to a standstill, it
is necessary that these immediate premises be indemonstrable. (APo. 72b21--23)
\end{quote-b}

Specifically, as we've seen above, Aristotle believed that we reliably gained
knowledge of the world via the senses, which then allowed for the subsumption of
particulars known via the senses under universals (categories, kinds, attributes),
which itself allowed for propositional knowledge of the kind found in syllogistic
argument. We'll discuss some of these points below.

The epistemological question of whether science is demonstrative, and whether there
is non-demonstrative knowledge of first principles, becomes a major issue in the 17th
and 18th centuries. We'll see this especially in our study of Descartes and Locke.
Central exemplars of demonstrative science in the early modern period include
geometry and basic number theory. Given this basis a major question in the 17th and
18th centuries concerned whether there was a demonstrative science of nature
(physics) and whether there was a demonstrative science of morality. Attempts to
answer these questions by various philosophers in the early modern period will occupy
us throughout the semester.

\section{Cosmology}
\label{sec:orge835fbe}
So far we've seen that, for Aristotle and the medieval philosophers influenced by
him, the world is construed as consisting of combinations of form and matter, proper
scientific knowledge of which requires an investigation of the cause(s) of a thing,
and is articulated in terms of demonstrative syllogisms whose fundamental premises
are known, at least ultimately, in a non-demonstrative manner. Non-demonstrative
knowledge is gained, as least partly, by our sensory experience of the natural world.
This conception of scientific knowledge resulted in a particular theory of the
structure of the universe (\emph{kosmos}) and our place in it and was heavily influenced by
christian doctrine. It was relatively stable until near the end of the Renaissance,
when Copernicus, Galileo, Kepler, and others argued that the Earth and other planets
and stars moved through space, and via imperfect (i.e. elliptical) orbits
(\citeprocitem{3}{Kuhn 1957}).

Before \href{http://plato.stanford.edu/entries/copernicus/}{Copernicus}, the dominant geocentric model was that articulated by \href{http://en.wikipedia.org/wiki/Ptolemy\#Astronomy}{Claudius
Ptolemy}, in which the entire universe rotated on a series of fixed `\href{http://en.wikipedia.org/wiki/Celestial\_spheres}{celestial
spheres}' in which were embedded all the stars and planets. The various 'spheres'
operated according to different laws. For example, material decay was a feature only
of matter in the `sublunary' sphere (i.e. in proximity to the Earth). The material
(the aether) which constituted the rest of the 'superlunary' world (i.e. the world
beyond the moon) was incorruptible. This idea that there were different laws for
different parts of the cosmos would be radically challenged by Descartes, Newton, and
Kant.

There were considered to be four fundamental qualities of things (related to
\href{http://plato.stanford.edu/entries/empedocles/}{Empedocles's} four elements), viz., \emph{hot}, \emph{cold}, \emph{wet}, and \emph{dry}. The combination of these
qualities gives rise to (and thus explain) the four elements (earth [cold, dry], air
[hot, wet], water [cold, wet], and fire [hot, dry]) and to all the other sensible
qualities of things (e.g. texture, color, odor, taste, etc.). Aristotle thus
considered hot, cold, wet, and dry the \emph{primary qualities} of things (i.e. substances).
Other qualities, such as texture are derivative of this primary qualities, and for
that reason classified as \emph{secondary} qualities. \href{http://plato.stanford.edu/entries/boyle/}{Robert Boyle} and \href{http://plato.stanford.edu/entries/locke/}{John Locke} would go
on to radically revise this model of explanation (see Locke's \href{http://www.earlymoderntexts.com/assets/pdfs/locke1690book2\_1.pdf}{\emph{Essay}, II.8}). They
would retain a primary/secondary distinction but revise the membership of these
categories.

Virtually every aspect of the medieval Aristotelian theory of the universe or cosmos
would be revised in some manner duering the early modern period. Copernicus's
displacement of the geocentric model of the cosmos with a heliocentric one is just
one of the earliest and clearest examples. And it is partly due to the displacement
of the standard view of the cosmos, heavily influenced as it was by Judeo-Christian
doctrine, that historians would come to mark this era as the beginning of the
`\href{https://en.wikipedia.org/wiki/Scientific\_revolution}{scientific revolution}.'



\section*{References}
\label{sec:org747bec9}
\begin{hangparas}{1.5em}{1}
\hypertarget{citeproc_bib_item_1}{De Jong, Willem R, and Arianna Betti. 2010. “The Classical Model of Science: A Millennia-Old Model of Scientific Rationality.” \textit{Synthese} 174 (2): 185–203.}

\hypertarget{citeproc_bib_item_2}{Jardine, Nicholas. 1988. “Epistemology of the Sciences.” In \textit{The Cambridge History of Renaissance Philosophy}, edited by Charles B. Schmitt, Quentin Skinner, Eckhard Kessler, and Jill Kraye, 685–711. Cambridge: Cambridge University Press.}

\hypertarget{citeproc_bib_item_3}{Kuhn, Thomas S. 1957. \textit{The Copernican Revolution: Planetary Astronomy in the Development of Western Thought}. Cambridge, MA: Harvard University Press.}

\hypertarget{citeproc_bib_item_4}{Randall, John Herman. 1961. \textit{The School of Padua and the Emergence of Modern Science}. Padova: Editrice Antenore.}

\hypertarget{citeproc_bib_item_5}{Shields, Christopher. 2007. \textit{Aristotle}. London: Routledge.}

\hypertarget{citeproc_bib_item_6}{———. 2015. “Aristotle.” In \textit{The Stanford Encyclopedia of Philosophy}, edited by Edward N Zalta.}
\end{hangparas}
\end{document}
