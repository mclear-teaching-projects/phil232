% Created 2022-01-25 Tue 10:21
% Intended LaTeX compiler: xelatex
\documentclass[12pt]{article}
                           
                           \input{/Users/roambot/.emacs.d/.local/custom-org-latex-classes/notes-setup-file.tex}
\author{PHIL 232}
\date{January 25, 2022}
\title{Locke \& Leibniz on Superaddition}
\makeatletter
\newcommand{\citeprocitem}[2]{\hyper@linkstart{cite}{citeproc_bib_item_#1}#2\hyper@linkend}
\makeatother

\usepackage[notquote]{hanging}
\begin{document}

\maketitle
\tableofcontents

\section{Mechanism \& Materialism}
\label{sec:org9fa84ed}

The seventeenth century saw the rise of two distinct but related positions: \emph{mechanism}
and \emph{materialism}. \emph{Materialism} is a metaphysical theory concerning the natural world.
It claims that the natural world consists fundamentally of a kind of \emph{thing} or
\emph{stuff}---matter---and that this stuff, organized in various ways according to natural
laws as characterized by physics, determines the nature and features of all the
objects we experience as part of the objective (and thus mind-independent) natural
world. Materialism says that there is fundamentally only one kind of thing---matter.
Materialism thus stands in opposition to a \emph{dualist} view like that articulated by
Descartes. Descartes argues that there are fundamentally \emph{two} kinds of thing---matter,
and mind---and that the natural world consists of these two kinds of thing
interacting with one another, sometimes in very special ways (as with the ensouled
bodies of human beings).

\emph{Mechanism} is part of a theory of \emph{explanation}. Lisa Downing, a scholar of the Early
Modern period, puts it this way,

\begin{quote-b}
[Mechanistic doctrine] states that all macroscopic bodily phenomena
should be explained in terms of the motions and impacts of
submicroscopic particles, or corpuscles, each of which can be fully
characterized in terms of a strictly limited range of (primary)
properties: size, shape, motion (or mobility), and, perhaps, solidity
or impenetrability. (\citeprocitem{4}{Downing 1998, 381})
\end{quote-b}

A 'corpuscle' is a extremely small parcel of matter with a determinate size, shape,
motion, location, etc. It is the features of individual corpuscles, plus their
interactions, which mechanism takes as sufficient for explaining the characteristics
and behavior of the natural (material) world. Thus, while the claim that the natural
world consists of nothing but corpuscles, or material particles, is a metaphysical
claim about what there is, mechanism is an explanatory claim concerning how appeal to
microscopic particles and their features is sufficient for explaining all natural
phenomena. These views thus compliment one another and often go together. But one
needn't be a materialist to endorse mechanism. For example, Descartes clearly
endorses a mechanist outlook on material nature.

\begin{quote-b}
I considered in general all the clear and distinct notions which our
understanding can contain with regard to material things. And I found no others
except for the notions we have of shapes, sizes and motions, and the rules in
accordance with which these three things can be modified by each other—rules
which are the principles of geometry and mechanics. And I judged as a result
that all the knowledge which men have of the natural world must necessarily be
derived from these notions. (\emph{Principles} 4:203; (\citeprocitem{3}{Descartes 1985}))
\end{quote-b}

One of the primary attractions of mechanistic explanation is that they seem
especially intelligible, given the assumption that material beings are understood
primarily as having properties that admit entirely of mathematical/geometric analysis
and explanation. 

In contrast to Descartes, who aimed primarily at a conception of material nature that
could in principle be entirely concieved through the basic or primary attribute of
extension, \href{https://plato.stanford.edu/entries/boyle}{Robert Boyle} and \href{https://plato.stanford.edu/entries/locke/}{John Locke} (both English philosophers) endorsed a
mechanism largely as a working explanatory hypothesis, that could be confirmed or
refuted on empirical grounds.

As Boyle put it,

\begin{quote-b}
These Principles, Matter, Motion (to which Rest is related) Bigness, Shape,
Posture, Order, Texture being so simple, clear, and comprehensive, are
applicable to all the real Phaenomena of Nature, which seem not to be explicable
by any other not consistent with ours. For, if recourse be had to an Immaterial
Principle or Agent, it may be such an one, as is not intelligible; and however
it will not enable us to explain the Phaenomena, because its way of working upon
things Material would probably be more difficult to be Physically made out, than
a Mechanical account of the Phaenomena. And, notwithstanding the Immateriality
of a created Agent, we cannot conceive, how it should produce changes in a Body,
without the help of Mechanical Principles, especially Local Motion
(\citeprocitem{1}{Boyle 1991, 153–54}).
\end{quote-b}

For Boyle, mechanism is simply the most explanatorily simple, clear, and
comprehensive theory we have regarding the material world, whether or not we think
material nature is all that there is (i.e. whether or not we endorse materialism).

Locke himself also frames his understanding of mechanism in terms of what he calls
the 'corpuscularian hypothesis'.

\begin{quote-b}
I have here instanced in the corpuscularian hypothesis, as that which
is thought to go furthest in an intelligible explication of those
qualities of bodies; and I fear the weakness of human understanding is
scarce able to substitute another, which will afford us a fuller and
clearer discovery of the necessary connexion and coexistence of the
powers which are to be observed united in several sorts of [bodies].
(ECHU IV.iii.16; (\citeprocitem{6}{Locke 1970}))
\end{quote-b}

There is a great deal to be said about both the doctrine of materialism
and that of mechanism, as well as their development and influence in the
seventeenth and eighteenth centuries. But here I focus on just one
issue---the specific properties corpuscles were thought to have, and the
explanatory role of these properties relative to all the other apparent
characteristics of objects.

Notice two things about mechanism. First, it ultimately concerns unobservable, or at
least, \emph{unobserved} entities (corpuscles). This means that the explanation of
observable phenomena depends on the existence and characteristics of unobserved
phenomena. This, in and of itself, is a dramatically anti-Aristotelian move. The
fundamental explanatory level of reality is one which is removed, and perhaps
ineluctably so, from our direct apprehension in experience.

Second, the explanatorily relevant features of corpuscles are taken by
mechanists to be their \emph{geometric} properties---viz., size, shape, location, and
state of motion. What are not included are those features which we might think
of as tied to specific ways of sensing the world---e.g. their colors, tastes, or
smells.\footnote{One feature---solidity---may or may not occupy a special role, being the
only proper sensible (i.e. sensory quality tied to a specific sense
modality, in this case touch) that may also be essential for the
mechanist. Locke seems especially concerned to see solidity as an
explanatorily essential quality, despite it not being strictly geometric
in nature.} This bifurcation in explanatory role meant both that greater
emphasis was placed on our \emph{mathematical} understanding of the natural world, and
that our purely sensory grasp of the natural world no longer played a
significant role (perhaps no role at all) in telling us how and why the world
appears to us as it does. To see this this shift to the characteristically
modern version of the ``primary/secondary'' quality distinction at work, it helps
to start with Galileo.

\section{Locke's Inconsistent Triad}
\label{sec:org75de4cc}

Locke appears to hold three claims that are not all compatible. First, he endorses
what I called, in the discussion of materialism and mechanism, the ``corpuscular
hypothesis''. This states that matter (or ``body'') fundamentally consists of only the
mechanical properties of shape, size, motion, and solidity.

\begin{quote-b}
Qualities thus considered in bodies are, First, such as are utterly inseparable from
the body, in what state soever it be; and such as in all the alterations and changes
it suffers, all the force can be used upon it, it constantly keeps; and such as sense
constantly finds in every particle of matter which has bulk enough to be perceived;
and the mind finds inseparable from every particle of matter, though less than to
make itself singly be perceived by our senses: v.g. Take a grain of wheat, divide it
into two parts; each part has still solidity, extension, figure, and mobility: divide
it again, and it retains still the same qualities; and so divide it on, till the
parts become insensible; they must retain still each of them all those qualities
(ECHU II.viii.9; (\citeprocitem{6}{Locke 1970})).
\end{quote-b}

\begin{quote-b}
Secondly, [there are qualities] which in truth are nothing in the objects themselves
but powers to produce various sensations in us by their primary qualities, i.e. by
the bulk, figure, texture, and motion of their insensible parts, as colours, sounds,
tastes, \&c. These I call secondary qualities (II.viii.10, 14, 23).
\end{quote-b}

According to Locke (following people such as Galileo and Boyle) body consists only of
mechanical qualities. All other other putative qualities, such as those of color or
smell, are simply powers that the mechanical qualities have to produce sensory ideas
in beings like ourselves.

Locke also holds that things have essences (or what he often calls ``real essences'').

\begin{quote-b}
Essence may be taken for the very being of anything, whereby it is what it is. And
thus the real internal, but generally (in substances) unknown constitution of things,
whereon their discoverable qualities depend, may be called their essence. This is the
proper original signification of the word, as is evident from the formation of it;
essentia, in its primary notation, signifying properly, being. And in this sense it
is still used, when we speak of the essence of particular things, without giving them
any name. (ECHU III.iii.15)
\end{quote-b}

In this Locke agrees with Descartes. The essence of thing determines what it is and
is referred to in explaining all of its other properties, or at least its
non-relational ones.

But Locke also seems to hold a third claim, that there is a kind of ``gap'' between the
existence of the mechanical properties of a body and other properties it has, such as
its ``secondary'' properties of color or smell.

\begin{quote-b}
But the coherence and continuity of the parts of Matter; the production of Sensation
in us of Colours and Sounds, etc. by impulse and motion; nay, the original Rules and
Communication of Motion being such wherein we can discover no natural connexion with
any Ideas we have, we cannot but ascribe them to the arbitrary Will and good Pleasure
of the Wise Architect. (ECHU IV.iii.29)
\end{quote-b}

Locke claims here that there is no clear connection between or derivation from
mechanical properties such as size or position the other properties of bodies, such
as the cohesion of their parts, the sensory ideas they prompt in us (i.e. ideas of
color or smell), or the motion of bodies. Instead, the connection or derivation of
this properites must be due to the \textbf{arbitrary} will of God. 

In sum then Locke holds the following three claims: 


\begin{enumerate}
\item \emph{Boylean corpuscularianism}: Bodies fundamentally have only the
mechanical qualities: shape, size, motion, and solidity.
\item \emph{Essentialism}: The qualities of things are all explained by their real
essences, i.e. their fundamental features, plus their spatial relations.
\item \emph{Gappiness}: Not all of the features of bodies are explained by their
shape, size, motion, and solidity.
\end{enumerate}


The problem is that if is Locke ascribing a certain group of properties of bodies to
God's ``arbitrary Will and good Pleasure'' then this undermines his essentialism, and
specifically the position of Mechanism, that only mechanical properties are
explanatorily relevant. If that is so then mechanism is doomed as a general
explanatory claim. Contemporary scholar Margaret Wilson puts it this way:

\begin{quote-b}
. . . at first thought it might seem that Locke could consistently hold that a body's
powers to produce ideas flow naturally from its real essence, while also maintaining
that the ideas themselves are arbitrarily annexed to whatever motions of matter
habitually cause them. But of course this is not really the case. For it follows from
Locke's account that a body has its powers to produce ideas only because of the
divine acts of annexation. Therefore, . . . we find conflict with the official
position that there is in reality an a priori conceptual connection between a body's
real essence and its secondary qualities. (\citeprocitem{7}{Wilson 1979, 147})
\end{quote-b}

Locke's endorsement of Gappiness undermines his conception of mechanistic explanation
through appeal to the essential features of bodies. But it does have one upside. It
allows him to possibly avoid the substance dualism and problems of causal interaction
confronted by Descartes. We see this in Locke's discussion of ``thinking matter'', as
matter that has properties of thinking ``superadded'' to it by God.

\begin{quote-b}
We have the ideas of matter and thinking, but possibly shall never be able to know
whether any mere material being thinks or no; it being impossible for us, by the
contemplation of our own ideas, without revelation, to discover whether Omnipotency
has not given to some systems of matter, fitly disposed, a power to perceive and
think, or else joined and fixed to matter, so disposed, a thinking immaterial
substance: it being, in respect of our notions, not much more remote from our
comprehension to conceive that GOD can, if he pleases, superadd to matter a faculty
of thinking, than that he should superadd to it another substance with a faculty of
thinking; since we know not wherein thinking consists, nor to what sort of substances
the Almighty has been pleased to give that power, which cannot be in any created
being, but merely by the good pleasure and bounty of the Creator. Whether Matter may
not be made by God to think is more than man can know. (ECHU IV.iii.6)
\end{quote-b}

Locke thus holds that God could have (and may indeed actually have) made it the
case that matter (i.e. extended and impenetrable substance) is endowed with the
property of thought, as ``superadded'' to it by God's act. Call this idea that
there can be thinking matter the claim of ``superaddition''.

\begin{description}
\item[{Superaddition:}] At least some properties of a substance are (or can be) explained
by appeal to God's will rather than the essence of the substance
\end{description}


The difficulty is that Gappiness and Superaddition radically undermine Locke's
otherwise more traditional conception of reality as consisting of substances with
essences that explain their properties and which come to be known to us through
experience, the knowledge of which is formalized through scientific theorizing.
Leibniz gives a clear articulation of this problem, to which we turn in the next
section.

\section{Leibniz's Objection}
\label{sec:orgf0ab4d3}

According to \href{https://plato.stanford.edu/entries/leibniz/}{Leibniz}, the conception of substance and essence requires that all
powers of objects are grounded in the nature of the objects themselves or in God's
activity of miraculous intervention. There cannot be non-miraculous ``superaddition''
of properties to a substance that do not follow from its essence.

\begin{quote-b}
one must above all take into account that the modifications which
can come naturally or without miracle to a single subject must come
to it from the limitations or variations of a real genus or of an
original nature, constant and absolute. For this is how in
philosophy we distinguish the modes of an absolute being from the
being itself; \ldots{} And every time we find some quality in a subject,
we ought to think that, if we understood the nature of this subject
and of this quality, we would understand how this quality could
result from that nature. Thus in the order of nature (setting
miracles aside) God does not arbitrarily give these or those
qualities indifferently to substances; he never gives them any but
those which are natural to them, that is to say, those that can be
derived from their nature as explicable modifications. \ldots{} This
distinction between what is natural and explicable and what is
inexplicable and miraculous removes all the difficulties: if we were
to reject it, we would uphold something worse than occult qualities,
and in doing so we would renounce philosophy and reason, and throw
open refuges for ignorance and idleness through a hollow system, a
system which admits not only that there are qualities we do not
understand (of which there are only too many) but also that there
are some qualities that the greatest mind could not understand, even
if God provided him with every possible advantage, that is,
qualities that would be either miraculous or without rhyme or
reason.(\citeprocitem{5}{Leibniz 1989, 304–5})
\end{quote-b}

Leibniz hammers away at the point that the reason for accepting a substance-essence
ontology is fundamentally one concerning \emph{explanation}. The idea being that reality
is, at least in principle, intelligible, in the sense that the ultimate explanation
of some property instantiation depends on appealing to the essence or nature of the
substance that has that property. Leibniz points out that once this connection
between substantial essence and property is rejected we no longer have any basis for
construing reality as in principle intelligible to us (or to anyone really, apart
from God). Here we see a fundamental difference between Leibniz's approach and
Locke's. Leibniz see reality as, in principle, fundamentally rationally intelligible,
while Locke either rejects its intelligibility or is at least deeply agnostic about
it.\footnote{See (\citeprocitem{2}{Connolly 2015}) for discussion of Locke's epistemic humility.}


\section*{References}
\label{sec:orgcb7a61c}
\begin{hangparas}{1.5em}{1}
\hypertarget{citeproc_bib_item_1}{Boyle, Robert. 1991. \textit{Selected Philosophical Papers of Robert Boyle}. Indianapolis: Hackett Publishing.}

\hypertarget{citeproc_bib_item_2}{Connolly, Patrick J. 2015. “Lockean Superaddition and Lockean Humility.” \textit{Studies in History and Philosophy of Science} 51: 53–61.}

\hypertarget{citeproc_bib_item_3}{Descartes, René. 1985. “Principles of Philosophy.” In \textit{1}, edited by John Cottingham, Robert Stoothoff, and Dugald Murdoch, 177–291. Cambridge: Cambridge University Press.}

\hypertarget{citeproc_bib_item_4}{Downing, Lisa. 1998. “The Status of Mechanism in Locke’s Essay.” \textit{The Philosophical Review} 107 (3): 381–414.}

\hypertarget{citeproc_bib_item_5}{Leibniz, Gottfried Wilhelm. 1989. “Preface to New Essays on Human Understanding.” In \textit{Philosophical Essays}, edited by Roger Ariew and Daniel Garber, 291–306. Hackett Publishing.}

\hypertarget{citeproc_bib_item_6}{Locke, John. 1970. \textit{An Essay Concerning Human Understanding}. Oxford: Oxford University Press.}

\hypertarget{citeproc_bib_item_7}{Wilson, Margaret D. 1979. “Superadded Properties: The Limits of Mechanism in Locke.” \textit{American Philosophical Quarterly} 16 (2): 143–50.}
\end{hangparas}
\end{document}
